\documentclass[]{beamer}
\usetheme{Dresden}
% \useoutertheme{split}

\usepackage{color}
\usepackage{graphicx}
\usepackage{listings}
\usepackage{lmodern} %% allow bold keywords
\usepackage{menukeys}
\usepackage{qtree}

\definecolor{darkgreen}{rgb}{0,0.5,0}
\definecolor{lightblue}{rgb}{0.2,0.2,1}

\lstset{language=Java,
	basicstyle=\ttfamily\footnotesize,
	keywordstyle=\color{purple},
	commentstyle=\color{darkgreen},
	numberstyle=\tiny\color{gray},
	stringstyle=\color{blue},
	tabsize=4,
	showstringspaces=false,
	breaklines=true,
	keepspaces=true,
	numbers=left,
	escapechar=@
}

\title{Java}
\subtitle{Control Structures}
\author{FSR Informatik}
\date{\today}

\begin{document}

\begin{frame}
\titlepage
\end{frame}
\begin{frame}{Overview}
\tableofcontents
\end{frame}

\section{Boolean Algebra}
\subsection{}
\begin{frame}[fragile]{Statements}
	A statement can be \textbf{true} or \textbf{false}. 
	You can use a boolean variable to save the value of a statement.
	\begin{lstlisting}
	boolean x1 = (17 < 20);  // true
	boolean x2 = (17 >= 20); // false
	boolean x3 = (17 == 20); // false
	boolean x4 = (17 != 20); // true
	\end{lstlisting}
\end{frame}

\begin{frame}{Relational Operators}
	Some relational operators for your statements:
	\begin{description}
		\item[A $<$ B] A smaller B
		\item[A $<=$ B] A smaller or equal B
		\item[A $>$ B] A greater B
		\item[A $>=$ B] A greater or euqal B
		\item[A $==$ B] A equal B
		\item[A $!=$ B] A not equal B
	\end{description}
\end{frame}

\begin{frame}{Logic}
	You can combine statements with logic operators like \textbf{NOT}, \textbf{AND} and \textbf{OR}.
	\begin{description}
		\item[!A] not A - is true if A is false
		\item[A \&\& B] A and B - is true if both statements are true
		\item[A \textbar\textbar\ B] A or B - is true if one statement is true or both
	\end{description}
\end{frame}

\begin{frame}[fragile]{Examples}
	\begin{lstlisting}
	boolean x1 = (17 < 20) && (4 < 16);  // true
	boolean x2 = (17 >= 20) || (4 < 16); // true
	boolean x3 = (17 == 20) && (4 == 4); // false
	boolean x4 = !(17 != 20);            // false
	boolean x5 = !(4 == 16) || (((17 == 20) && (4 == 4))); 
	// x5 is true
	\end{lstlisting}
\end{frame}

\section{Control Structures}
\subsection{Condition}
\begin{frame}[fragile]{If}
	\begin{lstlisting}
	public class PostOffice {
	
	    public static void main(String[] args) {
	
	        int letterWeight = 46; // in grams
	        int postage = 90; // in ct
	    
	        if(letterWeight <= 20) {
	            postage = 60;
	        }
	    }
	}
	\end{lstlisting}
\end{frame}

\begin{frame}[fragile]{If}
	%TODO red source code with same format as normal source code
	% if some is able to do that, pleas mail me
	If the \textcolor{red}{statement} is true the \textcolor{blue}{body} inside the curly brackets will be executed.
	If the \textcolor{red}{statement} is false the \textcolor{blue}{body} will not be executed.
	\begin{lstlisting}
	int letterWeight = 53; // in grams
	int postage = 90; // in ct
	    
	if(@\textcolor{red}{letterWeight <= 20}@) {
	    @\textcolor{blue}{postage = 60;}@
	}
	\end{lstlisting}
\end{frame}

\begin{frame}[fragile]{If - Example}
	\begin{lstlisting}
	public class PostOffice {
	
	    public static void main(String[] args) {
	
	        int letterWeight = 46;
	        int postage = 90;
	        
	        if(letterWeight <= 20) { // false
	            postage = 60;
	        }
	        
	        System.out.println(postage + "ct");
	        // prints: 90ct
	    }
	}
	\end{lstlisting}
	%\emph{If not neceassary the class head will not shown in the following examples.}
\end{frame}

\begin{frame}[fragile]{If - Counter Example}
	\begin{lstlisting}
	public class PostOffice {
	
	    public static void main(String[] args) {
	
	        int letterWeight = 17;
	        int postage = 90;
	    
	        if(letterWeight <= 20) { // true
	            postage = 60;
	        }
	        
	        System.out.println(postage + "ct");
	        // prints: 60ct
	    }
	}
	\end{lstlisting}
\end{frame}

\subsection{Else}
\begin{frame}[fragile]{Else}
	\begin{lstlisting}
	public class PostOffice {
	
	    public static void main(String[] args) {
	
	        int letterWeight = 17;
	        int postage = 0;
	    
	        if(letterWeight <= 20) { // true
	            postage = 60;
	        } else {
	            postage = 90;
	        }
	        
	        System.out.println(postage + "ct");
	        // prints: 60ct
	    }
	}
	\end{lstlisting}
\end{frame}

\begin{frame}[fragile]{Else If}
	\begin{lstlisting}
	public class PostOffice {
	
	    public static void main(String[] args) {
	
	        int letterWeight = 37;
	        int postage = 0;
	    
	        if(letterWeight <= 20) { // false
	            postage = 60;
	        } else if (letterWeight <= 50) { // true
	            postage = 90;
	        }
	        
	        System.out.println(postage + "ct");
	        // prints: 90ct
	    }
	}
	\end{lstlisting}
\end{frame}

\begin{frame}[fragile]{Multiple Else If}
	\begin{lstlisting}
	public static void main(String[] args) {
	
	    int letterWeight = 37;
	    int postage = 0;
	    
	    if(letterWeight <= 20) { // false
	        postage = 60;
	    } else if (letterWeight <= 50) { // true
	        postage = 90;
	    } else if (letterWeight <= 500 ) { // true
	        postage = 145;
	    }
	        
	    System.out.println(postage + "ct");
	    // prints: 90ct
	}
	\end{lstlisting}
\end{frame}

\begin{frame}{Multiple Else If}
	You can use as many \emph{else if} as you want.
	If multiple conditions are true, only the first one is relevant. \\
	\vfill
	\textbf{Warning: } Other programing languages may handle this case differently.
\end{frame}

\subsection{Loops}
\begin{frame}[fragile]{For Loop}
	The for loop starts with an assignment: \textcolor{gray}{\texttt{int i = 4}}. \\
	Every lap the \textcolor{blue}{body} will be executed and prints the changing variable i. \\
	After each lap the i will be incremented via \textcolor{orange}{\texttt{i++}}. \\
	The loop will stop if the condition \textcolor{red}{\texttt{i $<=$ 10}} becomes false. 
	It will never start if the condition is false at begin.
	\begin{lstlisting}
	public static void main(String[] args) {
	
	    for ( @\textcolor{gray}{int i = 4}@; @\textcolor{red}{i <= 10}@; @\textcolor{orange}{i++}@) {
	        @\textcolor{blue}{System.out.print(i + " ");}@   
	    }
	    //prints: 4 5 6 7 8 9 10
	}
	\end{lstlisting}
\end{frame}

\begin{frame}[fragile]{Endless Loop}
	If you need an endless loop. Use \textbf{for} with empty parameters.
	\begin{lstlisting}
	public static void main(String[] args) {
	
	    for (;;) {
	        System.out.println("I am still running");
	    }
	}
	\end{lstlisting}
\end{frame}

\begin{frame}[fragile]{While Loop}
	The while loop will be executed \emph{while} the \textcolor{red}{condition} is true.
	\begin{lstlisting}
	public static void main(String[] args) {
	
	    int i = 1;
	    while (@\textcolor{red}{i < 5}@) {
	        i++;
	        System.out.print(i + " ");
	    }
	    // prints: 2 3 4 5
	}
	\end{lstlisting}
\end{frame}

\begin{frame}[fragile]{Do-While Loop}
	The do-while loop will be executed until the \textcolor{red}{condition} becomes false.
	\begin{lstlisting}
	public static void main(String[] args) {
	
	    int i = 1;
	    do {
	        i++;
	        System.out.print(i + " ");
	    } while (@\textcolor{red}{i < 5}@);
	    // prints: 2 3 4 5
	}
	\end{lstlisting}
	\emph{Do not forget the semicolon at the end.}
\end{frame}

\begin{frame}{While vs. Do-While}
	There is a difference between the while and the do-while loop. \\
	\vfill
	If the loop condition false at start:
	\begin{itemize}
		\item the while loop will not start at all
		\item the do-while loop will run one time, if the condition stays false
	\end{itemize}
\end{frame}

\subsection{? Operator}
\begin{frame}{? Operator}
	\texttt{\textcolor{orange}{condition} \textcolor{red}{?} \textcolor{blue}{case1} 
		\textcolor{red}{:} \textcolor{gray}{case2} ;}
	\vfill
	If the \textcolor{orange}{condition} is true \textcolor{blue}{case1} will be executed.
	If not \textcolor{gray}{case2} will be executed instead.
\end{frame}
\begin{frame}[fragile]{? Operator - Example}
	Both methods do the same.
	\vfill
	\begin{lstlisting}
	public String boolToString1(boolean blub) {
	    return blub ? "yes" : "nope";
	}
	
	public String boolToString2(boolean blub) {
	    if (blub) {
	        return "yes";
	    } else { 
	        return "nope";
	    }
	}
	\end{lstlisting}
\end{frame}

\end{document}