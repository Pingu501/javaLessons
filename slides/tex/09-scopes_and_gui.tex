\input{../templates/course_definitions}
% This Document contains the information about this course.

% Authors of the slides
\author{Max Langer, Leonard Follner, Alexander Hesse}

% Name of the Course
\institute{Java-Kurs}

% Fancy Logo 
\titlegraphic{\hfill\includegraphics[height=1.25cm]{../templates/fsr_logo_cropped}}



% Custom Bindings
% \newcommand{\codeline}[1]{
%	\alert{\texttt{#1}}
%}


\title{Java}
\subtitle{Scopes \& GUI}
\date{\today}

\begin{document}

\begin{frame}
\titlepage
\end{frame}

\begin{frame}{Overview}
\tableofcontents
\end{frame}

\section{Scopes}
\subsection{Classes}

\begin{frame}[fragile]{Visiabilities}
	\begin{lstlisting}
	class MyGreatClass {
	
		//Attributes are public by default
		Car myCar; 
	
		//Public are available in every part of our code.
		public 
		
		//Private Attributes can only be acced via a method
		private House myHouse;
	}
		
	\end{lstlisting}
\end{frame}

\subsection{Controll Structures}
\begin{frame}[fragile]{If}
	\begin{lstlisting}
	...
	int a = 1;
	if(...) {
		int b = 5;
		System.out.println(a);
		System.out.println(b);
	}
	
	System.out.println(a);
	System.out.println(b);
	...
	\end{lstlisting}
	
	b is only available in the scope of the if.
	
	b is not outside of the if available.
	
	\color{red} WILL NOT COMPILE
\end{frame}

\begin{frame}[fragile]{For}
	\begin{lstlisting}
	for(int i = 0; i <= 100; i++) {
		int b = 3;
		System.out.println(i);
		System.out.println(b);
	}
	\end{lstlisting}
	
	b will be redefined in every round of the loop and is only available in the for loop.
	
	The scope is created at the begining and destroyed at the end of each round.
\end{frame}

\begin{frame}[fragile]{While}
	\begin{lstlisting}
	int i = 0;
	while(i <= 100) {
		int b = 3;
		System.out.println(i);
		System.out.println(b);
	}
	\end{lstlisting}
	
	For and while got the same scope behavior.
\end{frame}

\subsection{Mulit definitions}
\begin{frame}[fragile]{Examples}
	\begin{lstlisting}
	public class myClass {
		private int a;
		
		public myClass(int a) {
			this.a = a;
		}
	}
	\end{lstlisting}
	
	Use nearest definition.
	
	In one scope every variable name can be defined only one time.
\end{frame}

\begin{frame}{What we learned}
	Scopes are definition areas for variables.
	Every Block defines a new Scope.
\end{frame}


\end{document}