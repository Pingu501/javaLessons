\documentclass[]{beamer}
\usetheme{Dresden}
% \useoutertheme{split}

\usepackage{color}
\usepackage{graphicx}
\usepackage{listings}
\usepackage{lmodern} %% allow bold keywords
\usepackage{menukeys}
\usepackage{qtree}

\definecolor{darkgreen}{rgb}{0,0.5,0}
\definecolor{lightblue}{rgb}{0.2,0.2,1}

\lstset{language=Java,
	basicstyle=\ttfamily\footnotesize,
	keywordstyle=\color{purple},
	commentstyle=\color{darkgreen},
	numberstyle=\tiny\color{gray},
	stringstyle=\color{blue},
	tabsize=4,
	showstringspaces=false,
	breaklines=true,
	keepspaces=true,
	numbers=left,
	escapechar=@
}

\title{Java}
\subtitle{Controll Statements \& OOP}
\author{FSR Informatik}
\date{\today}

\tikzset{
  every overlay node/.style={
    draw=black,fill=white,rounded corners,anchor=north west,
  },
}
% Usage:
% \tikzoverlay at (-1cm,-5cm) {content};
% or
% \tikzoverlay[text width=5cm] at (-1cm,-5cm) {content};
\def\tikzoverlay{%
   \tikz[baseline,overlay]\node[every overlay node]
}%

\lstset{language=Java,
	basicstyle=\ttfamily\footnotesize,
	keywordstyle=\color{purple},
	commentstyle=\color{darkgreen},
	numberstyle=\tiny\color{gray},
	stringstyle=\color{blue},
	tabsize=4,
	showstringspaces=false,
	breaklines=true,
	keepspaces=true,
	numbers=left,
	escapechar=§
}

\begin{document}

\begin{frame}
\titlepage
\end{frame}
\begin{frame}{Overview}
\tableofcontents
\end{frame}

\section{Recalling last session}
\begin{frame}{Conclusion}
	Datatypes
	\begin{itemize}
		\item int, long
		\item float, double
		\item String
	\end{itemize}
	Hello World example
\end{frame}

\section{Controll Statements}
\begin{frame}{Controll Statements}
	
	\begin{itemize}
		\item if, else, else if
		\item for
		\item while
	\end{itemize}
		
\end{frame}

\subsection{Ite}
\begin{frame}[fragile]{{\huge I}f {\huge T}hen {\huge E}lse}
\begin{lstlisting}
if(condition) {
	//do something if condition is true
} else if(another condition){
	//do if "else if" condition is true 
} else {
	//otherwise do this
}
\end{lstlisting}
\end{frame}

\begin{frame}[fragile]{{\huge I}f {\huge T}hen {\huge E}lse example}
\begin{lstlisting}
public class IteExample {
	public static void main(String[] args) {
		int myNumber = 5;
		
		if(myNumber == 3) {
			System.out.println("Strange number");
		} else if(myNumber == 2) {
			System.out.println("Unreachable code");
		} else {
			System.out.println("Will be printed");
		}
	}
}
\end{lstlisting}
\end{frame}

\subsection{for}
\begin{frame}[fragile]{for}
\begin{lstlisting}
for(initial value, condition, change) {
	//do code while condition is true
}
\end{lstlisting}
\end{frame}

\begin{frame}[fragile]{for example}
\begin{lstlisting}
public class ForExample {
	public static void main(String[] args) {
		for(int i = 0; i <= 10; i++) {
			System.out.print("na ");
		}
		System.out.println("BATMAN!");
	}
}
\end{lstlisting}
\end{frame}

\subsection{while}
\begin{frame}[fragile]{while}
\begin{lstlisting}
while(condition) {
	//do code while condition is true
}
\end{lstlisting}
\end{frame}

\begin{frame}[fragile]{while example}
\begin{lstlisting}
public class WhileExample {
	public static void main(String[] args) {
		int a = 0;
		while(a <= 10) {
			System.out.println(a);
		}
	}
}
\end{lstlisting}
\end{frame}


\section{OOP in Java}

\begin{frame}{}
	\begin{center}
		{\huge Object Oriented Programming}
	\end{center}
\end{frame}

\subsection{General information}

\begin{frame}[fragile]{Class \emph{Student}}
\begin{lstlisting}
public class Student {

	// Attributes
	private String name; 
	private int matriculationNumber; 
	
	
	//Methods
	public void setName(String name) {
		this.name = name;
	}
	public int getMatriculationNumber() {
		return matriculationNumber;
	}
}
\end{lstlisting}

% What is visible here:
% Attributes store the state of the object
% Methods implement the behaviour of the object


\end{frame}

\begin{frame}[fragile]{Creation}
	We learned how to declare and assign a primitive datatype.
	
	\begin{lstlisting}
	    int a; // declare a
	    a = 273; // assign 273 to a
	\end{lstlisting} 
	
	The creation of an object works similar.
	
	\begin{lstlisting}
	    Student example = new Student(); 
		// create an instance of Student
	\end{lstlisting}
	The \textbf{object} derived from a \textbf{class} is also called \textbf{instance}.
	The variable is called the \textbf{reference}.
\end{frame}

\subsection{Methods}
\begin{frame}[fragile]{Calling a Method}
	\begin{lstlisting}
	public class Student {
		private String name;
	
		public String getName() {
        	return name;
    	}
		
	    public void setName(String newName) {
			name = newName;
	    }
	   
	}
	\end{lstlisting}
	The class \emph{Student} has two methods: \emph{void printTimetable()} and \emph{void printName()}.
\end{frame}

\begin{frame}[fragile]{Calling a Method}
\begin{lstlisting}
public class Main {
    public static void main(String[] args) {
        Student example = new Student(); // creation
        example.setName("Jane"); // method call
        String name = example.getName(); 
		System.out.println(name); // Prints "Jane"
    }
}
	\end{lstlisting}
	You can call a method of an object after its creation with \textbf{reference.methodName();}.
\end{frame}

\begin{frame}[fragile]{Calling a Method}
\begin{lstlisting}
	public class Student {
		private String name;
	
	    public void setName(String newName) {
			name = newName;
			printName();   // Call own method
			this.printName(); // Or this way
	    }
	    
	    public void printName() {
	        System.out.println(name);
	    }
	}
	\end{lstlisting}
	You can call a method of the own object by simply writing \textbf{methodName();} or \textbf{this.methodName();}
\end{frame}

\begin{frame}[fragile]{Methods with Arguments}

\begin{lstlisting}
public class Calc {
    public void add(int summand1, int summand2) {
        System.out.println(summand1 + summand2);
    }
	    
    public static void main(String[] args) {
        int summandA = 1;
        int summandB = 2;
        Calc calculator = new Calc();
        System.out.print("1 + 2 = ");
        calculator.add(summandA, summandB); 
		// prints: 3
    }
}
	\end{lstlisting}
\end{frame}

\subsection{Return Value}
\begin{frame}[fragile]{Methods with Return Value}
	A method without a return value is indicated by \textbf{void}:
	\begin{lstlisting}
	public void add(int summand1, int summand2) {
	    System.out.println(summand1 + summand2);
	}
	\end{lstlisting}
	A method with an \textbf{int} as return value:
	\begin{lstlisting}
	public int add(int summand1, int summand2) {
	    return summand1 + summand2;
	}
	\end{lstlisting}
	% TODO explain return statement
\end{frame}

\begin{frame}[fragile]{Calling Methods with a return value}
	\begin{lstlisting}
	public class Calc {
	
	    public int add(int summand1, int summand2) {
	        return summand1 + summand2;
	    }
	    
	    public static void main(String[] args) {
	        Calc calculator = new Calc();
	        int sum = calculator.add(3, 8);
	        System.out.print("3 + 8 = " + sum); 
			// prints: 3 + 8 = 11
	    }
	}
	\end{lstlisting}
\end{frame}

\subsection{Constructor}

\begin{frame}[fragile]{Constructors}
	\begin{lstlisting}
	public class Calc {
		private int summand1;
		private int summand2;
	
	    public Calc() {
			summand1 = 0;
			summand2 = 0;
	    }
	}
	\end{lstlisting}
	A constructor gets called upon creation of the object
\end{frame}

\begin{frame}[fragile]{Constructors with Arguments}
	\begin{lstlisting}
	public class Calc {
		private int summand1;
		private int summand2;
	
	    public Calc(int x, int y) {
			summand1 = x;
			summand2 = x;
	    }
	}
	[...]
	Calc myCalc = new Calc(7, 9);
	\end{lstlisting}
	A constructor can have Arguments aswell!
\end{frame}

\section{Conclusion}
\subsection{An Example}

\begin{frame}{An Example}
	You want to program an enrollment system, for a programming course. \\
	\vspace{1em}
	Your classes are:\\
	\begin{description}
		\item[student] who wants to attend the course
		\item[lesson] which is a part of the course
		\item[tutor] the guy with the bandshirt
		\item[room] where your lessons take place
		\item[\dots]
	\end{description}
% 	The more you think about it, the more complex this program becomes.
% 	Focus on the relevant things.
%	Think about how the objects can be in relation, this will be discussed later
%	Show prepared classes in Java
\end{frame}


\begin{frame}[fragile]{Class \emph{Student}}
\begin{lstlisting}
public static void main(String[] args) {
	Student peter = new Student();
	peter.changeName("Peter");
}
\end{lstlisting}
\end{frame}

\end{document}