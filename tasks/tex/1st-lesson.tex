\documentclass[a4paper, 12pt]{scrartcl}
\usepackage{ucs}
\usepackage[utf8x]{inputenc}
\usepackage[T1]{fontenc}
\usepackage{titlesec}
 \usepackage{mathptmx}
 \usepackage[scaled=.90]{helvet}
 \usepackage{courier}

\title{Temprature}
\author{}
\date{}
\pagenumbering{Roman}


\begin{document}
\raggedright {
    \Huge {
        {\fontfamily{cmss}\selectfont
            Temprature
            \vspace{2cm}     
               
        }
    }
}
\framebox {
    \parbox {\textwidth}{
        \textbf{Schwierigkeit}: leicht \\
        \textbf{Voraussetzungen}: Umgang mit einer Java IDE \\
        \textbf{Lernziel}: Grundlegendes Rechnen
    }
}
\vspace{1cm}

 \section*{Lessions}
 Convert 20°C, 40°C and 120°C into fahrenheit. \\
 Print the results on the console.
 
 \section*{Hints}
 \begin{itemize}
     \item For the conversion: %http://de.wikipedia.org/wiki/Grad_Fahrenheit/
     \item Check your solution: 30°C = 86°F
\end{itemize}
 
\end{document}