\documentclass[12pt, a4paper]{book}
\usepackage[utf8x]{inputenc}
\usepackage[usenames, dvipsnames]{color}

\usepackage{url}

\usepackage{titlesec}
% to change titles font family
\usepackage{titling}


%%% declare fonts and set some formats
% fontspec to use non-latex with xetex
\usepackage{xunicode}
\usepackage{fontspec}
\usepackage{xltxtra}

\usepackage{listings}

% font declaration and title settings
\newfontfamily\headingfont[]{Arial}
\titleformat{\chapter}[display]
  {\huge\headingfont}{\chaptertitlename\ \thechapter}{20pt}{\Huge}
\titleformat*{\section}{\Large\headingfont\color{Orange}}
\titleformat*{\subsection}{\Large\headingfont}
\renewcommand{\maketitlehooka}{\headingfont}

\definecolor{Orange}{RGB}{236, 127, 55}
\definecolor{StringColor}{RGB}{114, 0, 0}
\definecolor{NumberColor}{RGB}{0, 70, 109}
\definecolor{CommentColor}{RGB}{0, 70, 0}
\definecolor{KeywordColor}{RGB}{191, 0, 64}

\pagenumbering{Roman}

\lstset{language=Java,
    basicstyle=\ttfamily\footnotesize,
    keywordstyle=\color{KeywordColor},
    commentstyle=\color{CommentColor},
    numberstyle=\tiny\color{NumberColor},
    stringstyle=\color{StringColor},
    tabsize=4,
    showstringspaces=false,
    breaklines=true,
    keepspaces=true,
    numbers=left,
    escapechar=@
}


\begin{document}
\chapter*{Temperatur}

\headingfont
\parbox {\textwidth}{
    \textbf{Schwierigkeit}: Leicht \\
    \textbf{Voraussetzungen}: Umgang mit einer Java IDE \\
    \textbf{Lernziele}: Grundlegendes Rechnen
}
\normalfont
\begin{center}
\line(1,0){500}
\end{center}
\vspace{1cm}

\section*{Aufgaben}
Rechne 20°C, 40°C und 120°C in Grad Fahrenheit um und gib die Ergebnisse auf der Konsole aus.

\section*{Hinweise}
\begin{itemize}
     \item Für die Umrechnung: 
         \url{http://de.wikipedia.org/wiki/Grad_Fahrenheit/}
     \item zur Kontrolle: 30°C = 86°F
\end{itemize}

\end{document}