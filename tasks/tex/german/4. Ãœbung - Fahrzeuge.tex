\documentclass[12pt, a4paper]{book}
\usepackage[utf8x]{inputenc}
\usepackage[usenames, dvipsnames]{color}

\usepackage{url}

\usepackage{titlesec}
% to change titles font family
\usepackage{titling}


%%% declare fonts and set some formats
% fontspec to use non-latex with xetex
\usepackage{xunicode}
\usepackage{fontspec}
\usepackage{xltxtra}

\usepackage{listings}

% font declaration and title settings
\newfontfamily\headingfont[]{Arial}
\titleformat{\chapter}[display]
  {\huge\headingfont}{\chaptertitlename\ \thechapter}{20pt}{\Huge}
\titleformat*{\section}{\Large\headingfont\color{Orange}}
\titleformat*{\subsection}{\Large\headingfont}
\renewcommand{\maketitlehooka}{\headingfont}

\definecolor{Orange}{RGB}{236, 127, 55}
\definecolor{StringColor}{RGB}{114, 0, 0}
\definecolor{NumberColor}{RGB}{0, 70, 109}
\definecolor{CommentColor}{RGB}{0, 70, 0}
\definecolor{KeywordColor}{RGB}{191, 0, 64}

\pagenumbering{Roman}

\lstset{language=Java,
    basicstyle=\ttfamily\footnotesize,
    keywordstyle=\color{KeywordColor},
    commentstyle=\color{CommentColor},
    numberstyle=\tiny\color{NumberColor},
    stringstyle=\color{StringColor},
    tabsize=4,
    showstringspaces=false,
    breaklines=true,
    keepspaces=true,
    numbers=left,
    escapechar=@
}

\begin{document}
\chapter*{Fahrzeuge}

\headingfont
\parbox {\textwidth}{
    \textbf{Voraussetzungen}: Vererbung \\
    \textbf{Zeithorizont}: 45 Minuten \\
    \textbf{Lernziele}: Praktischer Umgang mit Vererbungen und abstrakten Klassen
}

\normalfont
\begin{center}
\line(1,0){500}
\end{center}
\vspace{1cm}

\section*{Beschreibung}
Es gibt mehrere Arten von Fahrzeugen. Autos und Motorräder sind Beispiele. Elektroautos sind spezielle Autos. \\
Jedes Fahrzeug besitzt eine bestimmte Anzahl von Rädern und wird von einer Marke hergestellt.
Ob Steuern für ein Fahrzeug gezahlt werden müssen, wird mit einem \texttt{boolean} festgelegt, welcher standartmäßig auf \texttt{true} gesetzt ist.
Autos können einen Autopiloten besitzen. Bei Elektroautos wird die Kapazität der Batterie angegeben außerdem müssen keine Steuern gezahlt werden.

\section*{Aufgaben}
\begin{enumerate}
     \item Modelliere die Klassen \textit{Vehicle}, \textit{Car}, \textit{ElectricCar} und \textit{Motorcycle}.
     \item Füge den Klassen die oben beschriebenen Variablen sowie deren Getter Methoden hinzu.
     \item Konstruktoren jedes Fahrzeugs nehmen die Marke (String) und das Elektroauto zusätzlich die Kapazität entgegen.
     \item Die toString Methode sollte den Markenname und bei dem Elektroauto zusätzlich die Kapazität ausgeben.
     \item Erstelle in einer extra Klasse \textit{Garage} mit einer main-Methode, die ein Auto, Elektroauto und ein Motorrad erstellt. \
            Gib für jedes Fahrzeug den Namen, die Anzahl der Räder und die Steuerpflicht aus.
\end{enumerate}

\section*{Hinweise}
\begin{itemize}
    \item Die Anzahl der Räder sollte sich von außen nicht ändern lassen.
    \item Es sollte nicht möglich sein, Objekte der Klasse \textit{Vehicle} zu erstellen.
\end{itemize}
\end{document}