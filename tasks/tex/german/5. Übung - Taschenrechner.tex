\documentclass[12pt, a4paper]{book}
\usepackage[utf8x]{inputenc}
\usepackage[usenames, dvipsnames]{color}

\usepackage{url}

\usepackage{titlesec}
% to change titles font family
\usepackage{titling}


%%% declare fonts and set some formats
% fontspec to use non-latex with xetex
\usepackage{xunicode}
\usepackage{fontspec}
\usepackage{xltxtra}

\usepackage{listings}

% font declaration and title settings
\newfontfamily\headingfont[]{Arial}
\titleformat{\chapter}[display]
  {\huge\headingfont}{\chaptertitlename\ \thechapter}{20pt}{\Huge}
\titleformat*{\section}{\Large\headingfont\color{Orange}}
\titleformat*{\subsection}{\Large\headingfont}
\renewcommand{\maketitlehooka}{\headingfont}

\definecolor{Orange}{RGB}{236, 127, 55}
\definecolor{StringColor}{RGB}{114, 0, 0}
\definecolor{NumberColor}{RGB}{0, 70, 109}
\definecolor{CommentColor}{RGB}{0, 70, 0}
\definecolor{KeywordColor}{RGB}{191, 0, 64}

\pagenumbering{Roman}

\lstset{language=Java,
    basicstyle=\ttfamily\footnotesize,
    keywordstyle=\color{KeywordColor},
    commentstyle=\color{CommentColor},
    numberstyle=\tiny\color{NumberColor},
    stringstyle=\color{StringColor},
    tabsize=4,
    showstringspaces=false,
    breaklines=true,
    keepspaces=true,
    numbers=left,
    escapechar=@
}

\begin{document}
\chapter*{Taschenrechner}

\headingfont
\parbox {\textwidth}{
    \textbf{Voraussetzungen}: Switch Statements \\
    \textbf{Zeithorizont}: 30 Minuten \\
    \textbf{Lernziele}: Praktischer Umgang mit Switch
}

\normalfont
\begin{center}
\line(1,0){500}
\end{center}
\vspace{1cm}

\section*{Aufgaben}
\begin{enumerate}
     \item Modelliere die Klasse \textit{Calculator}, welche eine \texttt{main} Methode enthält.
     \item Frage in der Klasse den Benutzer erst nach 2 Zahlen (\texttt{integer}).
     \item Frage nun den Benutzer nach der Rechenoperation (+, -, * oder /)(\texttt{String}).
     \item Berechne anhand der gegebenen Rechenoperation das Ergebnis und gib es aus.
     \item [Z 1.] Implementiere eine Lösung zum Problem "Division durch Null".
     \item [Z 2.] Nutze eine Schleife für die Eingabe, um beliebig viele Zahlen miteinander zu verrechnen.
\end{enumerate}

\section*{Hinweise}
\begin{itemize}
    \item Hinweis zu den Nutzereingaben: 
    \item Fehler bei Nutzereingaben (welche zu Abstürzen führen), können ignoriert werden.
\end{itemize}
\end{document}