\documentclass[12pt, a4paper]{book}
\usepackage[utf8x]{inputenc}
\usepackage[usenames, dvipsnames]{color}

\usepackage{url}

\usepackage{titlesec}
% to change titles font family
\usepackage{titling}


%%% declare fonts and set some formats
% fontspec to use non-latex with xetex
\usepackage{xunicode}
\usepackage{fontspec}
\usepackage{xltxtra}

\usepackage{listings}

% font declaration and title settings
\newfontfamily\headingfont[]{Arial}
\titleformat{\chapter}[display]
  {\huge\headingfont}{\chaptertitlename\ \thechapter}{20pt}{\Huge}
\titleformat*{\section}{\Large\headingfont\color{Orange}}
\titleformat*{\subsection}{\Large\headingfont}
\renewcommand{\maketitlehooka}{\headingfont}

\definecolor{Orange}{RGB}{236, 127, 55}
\definecolor{StringColor}{RGB}{114, 0, 0}
\definecolor{NumberColor}{RGB}{0, 70, 109}
\definecolor{CommentColor}{RGB}{0, 70, 0}
\definecolor{KeywordColor}{RGB}{191, 0, 64}

\pagenumbering{Roman}

\lstset{language=Java,
    basicstyle=\ttfamily\footnotesize,
    keywordstyle=\color{KeywordColor},
    commentstyle=\color{CommentColor},
    numberstyle=\tiny\color{NumberColor},
    stringstyle=\color{StringColor},
    tabsize=4,
    showstringspaces=false,
    breaklines=true,
    keepspaces=true,
    numbers=left,
    escapechar=@
}

\begin{document}
\chapter*{Zweidimensional}

\headingfont
\parbox {\textwidth}{
    \textbf{Voraussetzungen}: Umgang mit Interfaces \\
    \textbf{Zeithorizont}: 30 Minuten \\
    \textbf{Lernziele}: Implementierung von Interfaces
}

\normalfont
\begin{center}
\line(1,0){500}
\end{center}
\vspace{1cm}

\section*{Beschreibung}
\begin{lstlisting}
	public interface TwoDimensional {
    
        public double pi =  3.14159f;
        
        public double getPerimeter();
        
        public double getArea();
        
	}
\end{lstlisting}

\section*{Aufgaben}
\begin{enumerate}
    \item Implementiere für die Klassen \textit{Square}, \textit{Circle} und \textit{Rectangle} das Interface \textit{TwoDimensional}.
    \item Schreibe Konstruktoren die geeignete Parameter, wie z.B. Radius, der Körper übernehmen.
    \item Nutze die Konstante \textit{pi} wo nötig.
\end{enumerate}
\end{document}