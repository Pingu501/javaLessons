\documentclass[12pt, a4paper]{book}
\usepackage[utf8x]{inputenc}
\usepackage[usenames, dvipsnames]{color}

\usepackage{url}

\usepackage{titlesec}
% to change titles font family
\usepackage{titling}


%%% declare fonts and set some formats
% fontspec to use non-latex with xetex
\usepackage{xunicode}
\usepackage{fontspec}
\usepackage{xltxtra}

\usepackage{listings}

% font declaration and title settings
\newfontfamily\headingfont[]{Arial}
\titleformat{\chapter}[display]
  {\huge\headingfont}{\chaptertitlename\ \thechapter}{20pt}{\Huge}
\titleformat*{\section}{\Large\headingfont\color{Orange}}
\titleformat*{\subsection}{\Large\headingfont}
\renewcommand{\maketitlehooka}{\headingfont}

\definecolor{Orange}{RGB}{236, 127, 55}
\definecolor{StringColor}{RGB}{114, 0, 0}
\definecolor{NumberColor}{RGB}{0, 70, 109}
\definecolor{CommentColor}{RGB}{0, 70, 0}
\definecolor{KeywordColor}{RGB}{191, 0, 64}

\pagenumbering{Roman}

\lstset{language=Java,
    basicstyle=\ttfamily\footnotesize,
    keywordstyle=\color{KeywordColor},
    commentstyle=\color{CommentColor},
    numberstyle=\tiny\color{NumberColor},
    stringstyle=\color{StringColor},
    tabsize=4,
    showstringspaces=false,
    breaklines=true,
    keepspaces=true,
    numbers=left,
    escapechar=@
}

% \usepackage{courier}
\begin{document}
\chapter*{Library - Part 1}

\headingfont
\parbox {\textwidth}{
    \textbf{Difficulty}: Easy - 45 minutes \\
    \textbf{Requirements}: Creating classes, initialising objects \\
    \textbf{Learning outcome}: Working with simple classes
}

\normalfont
\begin{center}
\line(1,0){500}
\end{center}
\vspace{1cm}

\section*{Description}
Books can be borrowed from a small library. Each book has a title and an ISBN. The library can store one book multiple times. 
As the library is small, it can only store 10 books.

\section*{Tasks}
\begin{enumerate}
     \item Write the classes \textit{Library} and \textit{Book}.
     \item Write a constructor for the class \textit{Book}, which takes title and ISBN as arguments.
     \item Implement a \textit{main Method} in the class \textit{Library}, which tests the \textit{Book}'s methods.
\end{enumerate}

\section*{Hints}
\begin{itemize}
    \item An ISBN consists of 13 characters. Use \texttt{long} or \texttt{String} instead of \texttt{int}.
    \item When assigning the number as \texttt{long}, an \texttt{L} is attached: \texttt{long var = 1234501234567890L}.
\end{itemize}

\end{document}